% !TEX program = pdflatex
\documentclass[12pt]{article}
 \usepackage[margin=1in]{geometry} 
\usepackage{amsmath,amsthm,amssymb,amsfonts, enumitem, fancyhdr, color, comment, graphicx, environ}
\pagestyle{fancy}
\setlength{\headheight}{65pt}
\newenvironment{problem}[2][Problem]{\begin{trivlist}
\item[\hskip \labelsep {\bfseries #1}\hskip \labelsep {\bfseries #2.}]}{\end{trivlist}}
\newenvironment{sol}
    {\emph{Solution:}
    }
    {
    \qed
    }
\specialcomment{com}{ \color{blue} \textbf{Comment:} }{\color{black}}
\NewEnviron{probscore}{\marginpar{ \color{blue} \tiny Problem Score: \BODY \color{black} }}
\usepackage[UTF8]{ctex}
\usepackage[version=4]{mhchem}
\usepackage{listings}
\lstset{language=Matlab}
\lhead{Name: 陈稼霖\\ StudentID: 45875852}
\rhead{CHEM1111 \\ General Chemistry II \\ Spring 2019 \\ Homework 4}
\begin{document}
\begin{problem}{16.10}
Ammonium hexachloroplatinate(IV), \ce{(NH4)2(PtCl6}), is one of the few sparingly soluble ammonium salts. Its $K_{sp}$ at $20^{\circ}$C is $5.6\times10^{-6}$. Compute its solubility in grams per liter of solution.
\end{problem}
\begin{sol}
The solubility equilibrium is
\begin{center}
\ce{(NH4)2(PtCl6)(s) <=>2NH4^+(aq) + PtCl6^{2-}(aq)}
\end{center}
and the expression for the solubility product is
\[
K_{sp}=[NH_4^+]^2[PtCl_6^{2-}]=5.6\times10^{-6}
\]
Suppose $s$ mol of \ce{(NH4)2(PtCl6}) is dissolved in $1$ L, the equilibrium concentration of \ce{NH4^+} and \ce{PtCl6^{2-}} will be $2s$ \textsc{m} and $s$ \textsc{m}, respectively. So we have
\[
K_{sp}=[NH_4^+]^2[PtCl_6^{2-}]=(2s)^2\cdot s=4s^3=5.6\times10^{-6}\Longrightarrow s=0.011
\]
Therefore, the solubility in grams per liter of \ce{(NH4)2(PtCl6)} solution is
\[
solubility (g\cdot L^{-1})=s\cdot M((NH_4)_2(PtCl_6))=0.011mol\cdot L^{-1}\times443.87g\cdot mol^{-1}=\uline{4.9g\cdot L^{-1}}
\]
\end{sol}

\begin{problem}{16.15}
At $100 ^{\circ}$C, water dissolves $1.8\times10^{-2}$ g of AgCl per liter. Compute the $K_{sp}$ of \ce{AgCl} at this temperature.
\end{problem}
\begin{sol}
The solubility equilibrium is
\begin{center}
\ce{AgCl(s) <=>Ag+(aq) + Cl-(aq)}
\end{center}
The concentration of \ce{Ag+} and \ce{Cl-} is
\[
[Ag^+]=[Cl^-]=\frac{solubility (g\cdot L^{-1}) of AgCl}{M(AgCl)}=\frac{1.8\times10^{-2}g\cdot L^{-1}}{143.32g\cdot mol^{-1}}=1.26\times10^{-4}mol\cdot L^{-1}
\]
Therefore, the $K_{sp}$ of \ce{AgCl} is
\[
K_{sp}=[Ag^+][Cl^-]=1.26\times10^{-4}\times1.26\times10^{-4}mol=\uline{1.6\times10^{-8}}
\]
\end{sol}

\begin{problem}{16.20}
Suppose $100.0$ mL of a $0.0010$ \textsc{m} \ce{CaCl2} solution is added to $50.0$ mL of a $6.0\times10^{-5}$ \textsc{m} \ce{NaF} solution at $25^{\circ}$C. Determine whether \ce{CaF2}(s) ($K_{sp}=3.9\times10^{-11}$) tends to precipitate from this mixture.
\end{problem}
\begin{sol}
Suppose the total volume of the mixed solution is the sum of volume of the \ce{CaCl2} solution and the \ce{NaF} solution, which is
\[
V=V_1+V_2=100.0mL+50.0mL=150.0mL=0.1500L
\]
The concentration of \ce{Ca^{2+}} and \ce{F-} in the solution after mix are
\begin{align*}
[Ca^{2+}]=&\frac{c_1(CaCl_2)V_1}{V}=\frac{0.0010mol\cdot L^{-1}\times0.1000L}{0.1500L}=6.67\times10^{-4}mol\cdot L^{-1}\\
[F^-]=&\frac{c_2(NaF)V_2}{V}=\frac{6.0\times10^{-5}mol\cdot L^{-1}\times0.0500L}{0.1500L}=2.0\times10^{-5}mol\cdot L^{-1}
\end{align*}
The solubility equilibrium of \ce{CaF2} is
\begin{center}
\ce{CaF2(s) <=>Ca^{2+}(aq) + 2F-(aq)}
\end{center}
So the reaction quotient is
\[
Q_{sp}=[Ca^{2+}][F^-]^2=6.67\times10^{-4}\times(2.0\times10^{-5})^2=2.7\times10^{-13}
\]
Since $Q_{sp}=2.7\times10^{-13}<3.9\times10^{-11}=K_{sp}$, the \ce{CaF2}(s) \uline{does not tend to precipitate} from this mixture.
\end{sol}

\begin{problem}{16.27}
The solubility product of nickel(II) hydroxide, \ce{Ni(OH)2}, at $25^{\circ}$C is $K_{sp}=1.6\times10^{-16}$.\\
(a) Calculate the molar solubility of \ce{Ni(OH)2} in pure water at $25^{\circ}$.\\
(b) Calculate the molar solubility of \ce{Ni(OH)2} in $0.100$ \textsc{m} \ce{NaOH}.
\end{problem}
\begin{sol}
\\(a) The solubility equilibrium of \ce{Ni(OH)2} is
\begin{center}
\ce{Ni(OH)2(s) <=>Ni^{2+}(aq) + 2OH-(aq)}
\end{center}
and the expression for the solubility product is
\[
K_{sp}=[Ni^{2+}][OH^-]^2=1.6\times10^{-16}
\]
Suppose the molar solubility of \ce{Ni(OH)2} in pure water is $s_1$ \textsc{m}, then the concentration of \ce{Ni^{2+}} and \ce{OH-} are $s_1$ \textsc{m} and $2s_1$ \textsc{m}, respectively. So we have
\[
K_{sp}=[Ni^{2+}][OH^-]^2=s_1\cdot(2s_1)^2=4s_1^3=1.6\times10^{-16}\Longrightarrow s_1=3.4\times10^{-6}
\]
Therefore, the molar solubility of \ce{Ni(OH)2} in pure water at $25^{\circ}$C is \uline{$3.4\times10^{-6}$ \textsc{m}}.\\
(b) The concentration of \ce{OH-} in $0.100$ \textsc{m} \ce{NaOH} is $0.100$ \textsc{m}. Suppose the molar solubility of \ce{Ni(OH)2} in $0.100$ \textsc{m} \ce{NaOH} is $s_2$ \textsc{m}, then the concentration of \ce{Ni^{2+}} is $s_2$ \textsc{m}. So we have
\[
K_{sp}=[Ni^{2+}][OH^-]^2=s_2\cdot(0.100)^2=1.6\times10^{-16}\Longrightarrow s_2=1.6\times10^{-14}
\]
Therefore, the molar solubility of \ce{Ni(OH)2} in $0.100$ \textsc{m} \ce{NaOH} at $25^{\circ}$C is \uline{$1.6\times10^{-14}$ \textsc{m}}.
\end{sol}

\begin{problem}{16.32}
Compare the molar solubility of \ce{Mg(OH)2} in pure water with that in a solution buffered at pH $9.00$.
\end{problem}
\begin{sol}
The solubility equilibrium of \ce{Mg(OH)2} is
\begin{center}
\ce{Mg(OH)2(s) <=>Mg^{2+}(aq) + 2OH-(aq)}
\end{center}
and the expression for the solubility product is
\[
K_{sp}=[Mg^{2+}][OH^-]^2=1.2\times10^{-11}
\]
In pure water, suppose the molar solubility of \ce{Mg(OH)2} is $s_1$ \textsc{m}, then the concentration of \ce{Mg^{2+}} and \ce{OH-} are $s_1$ \textsc{m} and $2s_1$ \textsc{m}, respectively. So we have
\[
K_{sp}=[Mg^{2+}][OH^-]^2=s_1\cdot(2s_1)^2=4s_1^3=1.2\times10^{-11}\Longrightarrow s_1=1.4\times10^{-4}
\]
In a solution buffered at pH $9.00$, the concentration of \ce{OH-} is
\[
[OH^-]=10^{-14.00+pH}mol\cdot L^{-1}=10^{-5}mol\cdot L^{-1}
\]
Suppose the molar solubility of \ce{Mg(OH)2} is $s_2$ \textsc{m}, then the concentration of \ce{Mg^{2+}} is $s_2$ \textsc{m}. So we have
\begin{gather*}
K_{sp}=[Mg^{2+}][OH^-]^2=s_2\cdot(10^{-5})^2=1.2\times10^{-11}\\
\Longrightarrow s_2=0.12>s_1=1.4\times10^{-4}
\end{gather*}
Therefore, \uline{the molar solubility of \ce{Mg(OH)2} in pure water is less than in a solution buffered at pH $9.00$}.
\end{sol}

\begin{problem}{16.38}
The organic compound 18­-crown­-6 (see preceding problem) also binds strongly with the alkali metal ions in methanol.
\begin{center}
\ce{K+ + 18-crown-6 <=> [complex]+}
\end{center}
In methanol solution the equilibrium constant is $1.41\times10^6$. A similar reaction with Cs 1 has an equilibrium constant of only $2.75\times10^{4}$. A solution is made (in methanol) containing $0.020mol\cdot L^{-1}$ each of \ce{K+} and \ce{Cs+}. It also contains $0.30mol\cdot L^{-1}$ of 18­-crown­-6. Compute the equilibrium concentrations of both the uncomplexed \ce{K+} and the uncomplexed \ce{Cs+}.
\end{problem}
\begin{sol}
The solubility equilibrium of \ce{K+} with 18-crown-6 is
\begin{center}
\ce{Cs+(aq) + 18-crown-6(aq) <=> Cs-crown+(aq)}, $K_{sp}=2.75\times10^4$
\end{center}
Because $K_{sp}$ of both two solubility equilibrium is very large, the equilibrium concentration of 18-crown-6 is approximately
\begin{align*}
[18-crown-6(aq)]=&[18-crown-6(aq)]_0-[K^+]_0-[Cs^+]_0\\
=&0.30mol\cdot L^{-1}-0.020mol\cdot L^{-1}-0.020mol\cdot L^{-1}=0.26mol\cdot L^{-1}
\end{align*}
Suppose the equilibrium concentration of \ce{K+} and \ce{Cs+} are $-x$ \textsc{m} and $y$ \textsc{m}, respectively, then
\begin{table}[h]
\centering
\begin{tabular}{cccccc}
& \ce{K+(aq)} & + & \ce{18-crown-6(aq)} & \ce{<=>} & \ce{K-crown+(aq)}\\
Equilibrium, \textsc{m} & $x$ & & $0.26$ & & $0.020-x$
\end{tabular}
\end{table}
\begin{table}[h]
\centering
\begin{tabular}{cccccc}
& \ce{Cs+(aq)} & + & \ce{18-crown-6(aq)} & \ce{<=>} & \ce{Cs-crown+(aq)}\\
Equilibrium, \textsc{m} & $y$ & & $0.26$ & & $0.020-y$
\end{tabular}
\end{table}
So the expression for the solubility products of the two equilibrium is
\begin{gather*}
K_{sp1}=\frac{[K-crown^+]}{[K^+][18-crown-6]}=\frac{0.020-x}{0.26x}=1.41\times10^6\\
K_{sp2}=\frac{[Sc-crown^+]}{[Cs^+][18-crown-6]}=\frac{0.020-y}{0.26y}=2.75\times10^{4}\\
\Longrightarrow x=5.5\times10^{-8},~~y=2.8\times10^{-6}
\end{gather*}
Therefore, the equilibrium concentrations of the uncomplexed \ce{K+} and umcomplexed \ce{Cs+} are \uline{$5.5\times10^{-8}mol\cdot L^{-1}$} and \uline{$2.8\times10^{-6}mol\cdot L^{-1}$}, respectively.
\end{sol}

\begin{problem}{16.47}
$K_{sp}$ for \ce{Pb(OH)2} is $4.2\times10^{-15}$, and $K_f$ for \ce{Pb(OH)3^-} is $4\times10^{14}$. Suppose a solution whose initial concentration of \ce{Pb^{2+}}(aq) is $1.00$ \textsc{m} is brought to pH $13.0$ by addition of solid \ce{NaOH}. Will solid \ce{Pb(OH)2} precipitate, or will the lead be dissolved as \ce{Pb(OH)3^-}(aq)? What will be \ce{Pb^{2+}} and \ce{[Pb(OH)3^-]} at equilibrium? Repeat the calculation for an initial \ce{[Pb^{2+}]} concentration of $0.050$ \textsc{m}. (Hint: One way to solve this problem is to assume that \ce{Pb(OH)2}(s) is present and calculate \ce{Pb^{2+}} and \ce{[Pb(OH)3^-]} that would be in equilibrium with the solid. If the sum of these is less than the original \ce{Pb^{2+}}, the remainder can be assumed to have precipitated. If not, there is a contradiction and we must assume that no \ce{Pb(OH)2}(s) is present. In this case we can calculate \ce{Pb^{2+}} and \ce{[Pb(OH)3^-]} directly from $K_f$.)
\end{problem}
\begin{sol}
The equilibrium equation and the expression of equilibrium equation of \ce{Pb(OH)3} and \ce{Pb} are
\begin{center}
\ce{Pb(OH)2(s) <=>Pb^{2+}(aq) + 2OH-(aq)}
\end{center}
\[
K_{sp}=[Pb^{2+}][OH^-]^2=4.2\times10^{-15}
\]
\begin{center}
\ce{Pb^{2+}(aq) + 3OH-(aq) <=>Pb(OH)3^-(aq)}
\end{center}
\[
K_f=\frac{[Pb(OH)_3^-]}{[Pb^{2+}][OH^-]^3}=4\times10^{14}
\]
In the solution with pH pf $13.0$, the concentration of \ce{OH-} is
\[
[OH^-]=10^{-14.0+pH}mol\cdot L^{-1}=10^{-14.0+13.0}mol\cdot L^{-1}=0.10mol\cdot L^{-1}
\]
\uline{When the initial concentration of $1.00$ \textsc{m}}: assume that \ce{Pb(OH)2}(s) is present, then we have
\[
K_{sp}=[Pb^{2+}][OH^-]^2=[Pb^{2+}]\times0.10^2=4.2\times10^{-15}\Longrightarrow \uline{[Pb^{2+}]=4.2\times10^{-13}}
\]
and
\[
K_f=\frac{[Pb(OH)_3^-]}{[Pb^{2+}][OH^-]^3}=\frac{[Pb(OH)_3^-]}{4.2\times10^{-13}\times0.10^3}=4\times10^{14}\Longrightarrow\uline{[Pb(OH)_3^-]=0.2mol\cdot L^{-1}}
\]
Since $[Pb^{2+}]+[Pb(OH)_3^-]=4.2\times10^{-13}mol\cdot L^{-1}+0.2mol\cdot L^{-1}<1.00mol\cdot L^{-1}=[Pb^{2+}]_0$, the assumption is correct. Therefore, \uline{solid  \ce{Pb(OH)2} precipitate and some lead will be dissolved as \ce{Pb(OH)3^-}}.\\
\uline{When the initial concentration of $0.050$ \textsc{m}}: the former assumption is not correct any more, since $[Pb^{2+}]+[Pb(OH)_3^-]=4.2\times10^{-13}+0.2mol\cdot L^{-1}>0.050mol\cdot L^{-1}=[Pb^{2+}]_0$. Therefore, \uline{solid \ce{Pb(OH)2} will not precipitate, but the lead will be dissolved as \ce{Pb(OH)3^-}}. In this way, suppose the concentration of \ce{Pb^{2+}} and \ce{Pb(OH)3^-} will have following relationship
\[
[Pb^{2+}]+[Pb(OH)_3^-]=[Pb^{2+}]_0=0.050mol\cdot L^{-1}
\]
So we have
\begin{gather*}
K_f=\frac{[Pb(OH)_3^-]}{[Pb^{2+}][OH^-]^3}=\frac{(0.050-[Pb^{2+}])}{[Pb^{2+}]\times0.10^3}=4\times10^{14}\\
\Longrightarrow\uline{[Pb^{2+}]=1\times10^{-13}mol\cdot L^{-1}},~~\uline{[Pb(OH)_3^-]=0.050mol\cdot L^{-1}}
\end{gather*}
\end{sol}

\begin{problem}{16.53}
Calculate the [\ce{Zn^{2+}}] in a solution that is in equilibrium with \ce{ZnS}(s) and in which $[H_3O^+]=1.0\times10^{-5}$ \textsc{m} and $[H_2S]=0.10$ \textsc{m}.
\end{problem}
\begin{sol}
The concentration of \ce{OH-} is
\[
[OH^-]=\frac{K_w}{[H_3O^+]}=\frac{10^{-14}}{1.0\times10^{-5}}mol\cdot L^{-1}=1.0\times10^{-9}mol\cdot L^{-1}
\]
The ionization equilibrium equation of \ce{H2S} is
\begin{center}
\ce{H2S(aq) + H2O(l) <=>H3O+(aq) + HS-(aq)}
\end{center}
and the expression of its ionization constant is
\[
K_a=\frac{[H_3O^+][HS^-]}{[H_2S]}=\frac{1.0\times10^{-5}\times[HS^-]}{0.10}=9.1\times10^{-8}\Longrightarrow[HS^-]=9.1\times10^{-4}mol\cdot L^{-1}
\]
The solubility equilibrium of \ce{ZnS} is
\begin{center}
\ce{ZnS(s) + H2O(l) <=>Zn^{2+}(aq) + HS-(aq) + OH-(aq)}
\end{center}
and the expression of its solubility product is
\begin{gather*}
K_{sp}=[Zn^{2+}][HS^-][OH^-]=[Zn^{2+}]\times9.1\times10^{-4}\times1.0\times10^{-9}=2\times10^{-25}\\
\Longrightarrow[Zn^{2+}]=\uline{2\times10^{-13}mol\cdot L^{-1}}
\end{gather*}
\end{sol}

\begin{problem}{16.56}
What is the highest pH at which $0.050$ \textsc{m} \ce{Mn^{2+}} will remain entirely in a solution that is saturated with \ce{H2S} at a concentration of $[H_2S]=0.10$ \textsc{m}? At this pH, what would be the concentration of \ce{Cd^{2+}} in equilibrium with solid \ce{CdS} in this solution?
\end{problem}
\begin{sol}
Suppose the concentration of \ce[H3O+] is $x$ \textsc{m}.Then the concentration of \ce{OH-} is
\[
[OH^-]=\frac{K_w}{[H_3O^+]}=\frac{10^{-14}}{x}mol\cdot L^{-1}
\]
The ionization equilibrium equation of \ce{H2S} is
\begin{center}
\ce{H2S(aq) + H2O(l) <=>H3O+(aq) + HS-(aq)}
\end{center}
and the expression of its ionization constant is
\[
K_a=\frac{[H_3O^+][HS^-]}{[H_2S]}=\frac{x[HS^-]}{0.10}=9.1\times10^{-8}\Longrightarrow[HS^-]=\frac{9.1\times10^{-9}}{x}mol\cdot L^{-1}
\]
The solubility equilibrium of \ce{MnS} is
\begin{center}
\ce{MnS(s) + H2O(l) <=>Mn^{2+}(aq) + HS-(aq) + OH-(aq)}
\end{center}
and the expression of its solubility product is
\begin{gather*}
K_{sp}=[Mn^{2+}][HS^-][OH^-]=0.050\times\frac{9.1\times10^{-9}}{x}\times\frac{10^{-14}}{x}=3\times10^{-14}\\
\Longrightarrow\uline[H_3O^+]=xmol\cdot L^{-5}=1.2\times10^{-5}mol\cdot L^{-1}
\end{gather*}
Therefore, the highest pH at which $0.050$ \textsc{m} \ce{Mn^{2+}} will remain entirely in a solution that is saturated with \ce{H2S} at a concentration of $[H_2S]=0.10$ \textsc{m} is
\[
\uline{pH}=-\log_{10}[H_3O^+]=-\log_{10}(1.2\times10^{-5})=\uline{4.92}
\]
At this pH, for the solubility equilibrium of solid \ce{CdS}, we have
\begin{gather*}
K_{sp}=[Cd^{2+}][HS^-][OH^-]=[Cd^{2+}]\times\frac{9.1\times10^{-9}}{1.2\times10^{-5}}\times\frac{10^{-14}}{1.2\times10^{-5}}=7\times10^{-28}\\
\Longrightarrow\uline{[Cd^{2+}]}=\uline{1.1\times10^{-15}mol\cdot L^{-1}}
\end{gather*}
\end{sol}
\end{document}