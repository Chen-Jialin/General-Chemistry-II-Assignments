% !TEX program = pdflatex
\documentclass[12pt]{article}
 \usepackage[margin=1in]{geometry} 
\usepackage{amsmath,amsthm,amssymb,amsfonts, enumitem, fancyhdr, color, comment, graphicx, environ}
\pagestyle{fancy}
\setlength{\headheight}{65pt}
\newenvironment{problem}[2][Problem]{\begin{trivlist}
\item[\hskip \labelsep {\bfseries #1}\hskip \labelsep {\bfseries #2.}]}{\end{trivlist}}
\newenvironment{sol}
    {\emph{Solution:}
    }
    {
    \qed
    }
\specialcomment{com}{ \color{blue} \textbf{Comment:} }{\color{black}}
\NewEnviron{probscore}{\marginpar{ \color{blue} \tiny Problem Score: \BODY \color{black} }}
\usepackage[UTF8]{ctex}
\usepackage[version=4]{mhchem}
\lhead{Name: 陈稼霖\\ StudentID: 45875852}
\rhead{CHEM1111 \\ General Chemistry II \\ Spring 2019 \\ Homework 10}
\begin{document}
\begin{problem}{18.6}
In the presence of vanadium oxide, \ce{SO2}(g) reacts with an excess of oxygen to give \ce{SO3}(g):
\begin{center}
\ce{SO2(g) + \frac{1}{2}O2(g) ->[V2O5] SO3(g)}
\end{center}
This reaction is an important step in the manufacture of sulfuric acid. It is observed that tripling the \ce{SO2} concentration increases the rate by a factor of $3$, but tripling the \ce{SO3} concentration decreases the rate by a factor of $1.7\approx\sqrt{3}$. The rate is insensitive to the \ce{O2} concentration as long as an
excess of oxygen is present.\\
(a) Write the rate expression for this reaction, and give the units of the rate constant $k$.\\
(b) If [\ce{SO2}] is multiplied by $2$ and [\ce{SO3}] by $4$ but all other conditions are unchanged, what change in the rate will be observed?
\end{problem}
\begin{sol}
\\(a) The rate expression for this reaction is
\[
\text{rate}=\uline{k[SO_2][SO_3]^{-\frac{1}{2}}}
\]
The units of the rate constant $k$ are \uline{mol L$^{-\frac{1}{2}}$ s$^{-1}$}.
\\(b) The rate will \uline{not} change.
\end{sol}

\begin{problem}{18.8}
The rate for the oxidation of iron(II) by cerium(IV)
\begin{center}
\ce{Ce^{4+}(aq) + Fe^{2+}(aq) -> Ce^{3+}(aq) + Fe^{3+}(aq)}
\end{center}
is measured at several different initial concentrations of the two reactants:
\begin{table}[h]
\centering
\begin{tabular}{ccc}
[\ce{Ce^4+}](mol L$^{-1}$) & [\ce{Fe^{2+}}](mol L$^{-1}$) & Rate(mol L$^{-1}$ s$^{-1}$) \\ \hline
$1.1\times10^{-5}$                                                   & $1.8\times10^{-5}$                                                   & $2.0\times10^{-7}$          \\
$1.1\times10^{-5}$                                                   & $2.8\times10^{-5}$                                                   & $3.1\times10^{-7}$          \\
$3.4\times10^{-5}$                                                   & $2.8\times10^{-5}$                                                   & $9.5\times10^{-7}$          \\ \hline
\end{tabular}
\end{table}
\\(a) Write the rate expression for this reaction.\\
(b) Calculate the rate constant $k$ and give its units.\\
(c) Predict the initial reaction rate for a solution in which [\ce{Ce^{4+}}] is $2.6\times10^{-5}$ \textsc{m} and [\ce{Fe^{2+}}] is $1.3\times10^{-5}$ \textsc{m}.
\end{problem}
\begin{sol}
\\(a) Suppose the rate expression for this reaction is
\[
\text{rate}=k[Ce^{4+}]^{n_1}[Fe^{2+}]^{n_2}
\]
\begin{gather*}
\frac{\text{rate}_2}{\text{rate}_1}=(\frac{[Fe^{2+}]_2}{[Fe^{2+}]_1})^{n_2}\Longrightarrow\frac{3.1\times10^{-7}}{2.0\times10^{-7}}=\frac{2.8\times10^{-5}}{1.8\times10^{-5}}\\
\Longrightarrow n_2=1\\
\frac{\text{rate}_3}{\text{rate}_2}=(\frac{[Ce^{4+}]_3}{[Ce^{4+}]_2})^{n_1}\Longrightarrow\frac{9.5\times10^{-7}}{3.1\times10^{-7}}=(\frac{3.4\times10^{-5}}{1.1\times10^{-5}})\\
\Longrightarrow n_1=1
\end{gather*}
Therefore, the rate expression for this reaction is
\[
\uline{\text{rate}=k[Ce^{4+}][Fe^{2+}]}
\]
(b)
\begin{gather*}
\text{rate}_1=k[Ce^{4+}]_1[Fe^{2+}]_1\Longrightarrow2.0\times10^{-7}=k\times1.1\times10^{-5}mol~L^{-1}\times1.8\times10^{-5}mol~L^{-1}\\
\Longrightarrow k=1.0\times10^3mol^{-1}~L~s^{-1}
\end{gather*}
Therefore, the rate constant $k$ is \uline{$1.0\times10^3$}, and its units are \uline{mol$^{-1}$ L s$^{-1}$}.\\
(c) The initial rate is
\[
\text{rate}=k[Ce^{4+}][Fe^{2+}]=1.0\times10^3\times2.6\times10^{-5}\times1.3\times10^{-5}mol~L^{-1}~s^{-1}=\uline{3.4\times10^{-7}mol L^{-1} s^{-1}}
\]
\end{sol}

\begin{problem}{18.17}
The rate for the reaction
\begin{center}
\ce{OH-(aq) + NH4^+(aq) -> H2O(l) + NH3(aq)}
\end{center}
is first order in both \ce{OH-} and \ce{NH4^+} concentrations, and the rate constant $k$ at $20^{\circ}$C is $3.4\times10^{10}$ L mol$^{-1}$ s$^{-1}$. Suppose $1.00$ L of a $0.0010$ \textsc{m} \ce{NaOH} solution is rapidly mixed with the same volume of $0.0010$ \textsc{m} \ce{NH4Cl} solution. Calculate the time (in seconds) required for the \ce{OH-} concentration to decrease to a value of $1.0\times{-5}$ \textsc{m}.
\end{problem}
\begin{sol}
The initial volume of \ce{OH-} and \ce{NH4^+} is
\[
[OH^-]_0=[NH_4^+]_0=\frac{0.0010mol~L^{-1}\times1.00L}{1.00L+1.00L}=0.00050mol~L^{-1}
\]
The rate for reaction can be written as
\begin{gather*}
\text{rate}=-\frac{d[OH^-]}{dt}=k[OH^-][NH_4^+]=k[OH^-]^2\\
\Longrightarrow\frac{1}{[OH^-]}-\frac{1}{[OH^-]_0}=kt
\Longrightarrow t=\frac{\frac{1}{[OH^-]}-\frac{1}{[OH^-]_0}}{k}=\frac{\frac{1}{1.0\times10^{-5}}-\frac{1}{0.00050}}{3.4\times10^{10}}s=\uline{2.9\times10^{-6}s}
\end{gather*}
\end{sol}

\begin{problem}{18.25}
Write the overall reaction and rate laws that correspond to the following reaction mechanisms. Be sure to eliminate intermediates from the answers.\\
(a) \ce{A + B <=>[k_1][k_{-1}] C + D}\hfill(fast equilibrium)\\
\indent\ce{C + E ->[k_2] F}\hfill(slow)\\
(b) \ce{A <=>[k_1][k_{-1}] B +C}\hfill(fast equilibrium)\\
\indent\ce{C + D <=>[k_2][k_{-2}] E}\hfill(fast equilibrium)\\
\indent\ce{E ->[k_3] F}\hfill(slow)
\end{problem}
\begin{sol}
\\(a) The overall reaction is
\begin{center}
\ce{A + B + E -> D + F}
\end{center}
The equilibrium constant of the first reaction is
\begin{gather*}
\frac{[C][D]}{[A][B]}=K=\frac{k_1}{k_{-1}}\\
\Longrightarrow[C]=\frac{k_1[A][B]}{k_{-1}[D]}
\end{gather*}
The rate law is
\[
\text{rate}=k_2[C][E]=\frac{k_1k_2[A][B][E]}{k_{-1}[D]}
\]
(b) The overall reaction is
\begin{center}
\ce{A + D -> B + F}
\end{center}
The equilibrium constant of the first and the second reaction is
\begin{gather*}
\frac{[B][C]}{[A]}=K_1=\frac{k_1}{k_{-1}}\Longrightarrow[C]=\frac{k_1[A]}{k_{-1}[B]}\\
\frac{[E]}{[C][D]}=K_2=\frac{k_2}{k_{-2}}\Longrightarrow[E]=\frac{k_2}{k_{-2}}[C][D]=\frac{k_1k_2[A][D]}{k_{-1}k_{-2}[B]}
\end{gather*}
The rate law is
\[
\text{rate}=k_3[E]=\frac{k_1k_2k_3[A][D]}{k_{-1}k_{-2}[B]}
\]
\end{sol}

\begin{problem}{18.33}
The mechanism for the decomposition of \ce{NO2Cl} is
\begin{center}
\ce{NO2Cl <=>[k_1][k_{-1}] NO2 + Cl}\\
\ce{NO2Cl + Cl ->[k_2] NO2 + Cl2}
\end{center}
By making a steady­state approximation for [\ce{Cl}], express the rate of appearance of \ce{Cl2} in terms of the concentrations of \ce{NO2Cl} and \ce{NO2}.
\end{problem}
\begin{sol}
The net rate of change of [\ce{Cl}] is
\begin{gather*}
\frac{d[Cl]}{dt}=k_1[NO_2Cl]-k_{-1}[NO_2][Cl]-k_3[NO_2Cl][Cl]=0\\
\Longrightarrow[Cl]=\frac{k_1[NO_2Cl]}{k_{-1}[NO_2]+k_2[NO_2Cl]}
\end{gather*}
The rate of the overall reaction \ce{2NO2Cl -> 2NO2 + Cl2} is
\[
\text{rate}=\frac{1}{2}\frac{d[Cl_2]}{dt}=k_2[NO_2Cl][Cl]=\frac{k_1k_2[NO_2Cl]^2}{k_{-1}[NO_2]+k_2[NO_2Cl]}
\]
\end{sol}

\begin{problem}{18.38}
Dinitrogen tetraoxide (\ce{N2O4}) decomposes spontaneously at
room temperature in the gas phase:
\begin{center}
\ce{N2O4(g) -> 2NO2(g)}
\end{center}
The rate law governing the disappearance of \ce{N2O4} with time is
\[
-\frac{d[N_2O_4]}{dt}=k[N_2O_4]
\]
At $30^{\circ}$C, $k=5.1\times10^6s^{-1}$ and the activation energy for the reaction is $54.0 kJ~mol^{-1}$.\\
(a) Calculate the time (in seconds) required for the partial pressure of \ce{N2O4(g)} to decrease from $0.10$ atm to $0.010$ atm at $30^{\circ}$C.
(b) Repeat the calculation of part (a) at $300^{\circ}$C.
\end{problem}
\begin{sol}
Integrate the reaction rate equation to get
\begin{gather*}
\ln[N_2O_4]-\ln[N_2O_4]_0=-kt\Longrightarrow\ln0.010-\ln0.10=-5.1\times10^6s^{-1}t\\
\Longrightarrow t=\uline{4.5\times10^7s}
\end{gather*}
(b) The rate constant at $300^{\circ}$ is
\begin{align*}
k'=&\frac{k}{e^{-\frac{E_a}{RT}}}e^{-\frac{E_a}{RT'}}\\
=&\frac{5.1\times10^6s^{-1}}{e^{-\frac{54.0\times10^3J~mol^{-1}}{8.31J~mol^{-1}~K^{-1}\times(30+273.15)K}}}e^{-\frac{54.0\times10^3J~mol^{-1}}{8.31J~mol^{-1}~K^{-1}\times(300+273.15)K}}\\
=&1.2\times10^{11}s^{-1}
\end{align*}
Therefore,
\begin{gather*}
\ln[N_2O_4]-\ln[N_2O_4]_0=-k't\Longrightarrow\ln0.010-\ln0.10=-1.2\times10^{11}s^{-1}t\\
\Longrightarrow t=\uline{1.9\times10^{-11}s}
\end{gather*}
\end{sol}

\begin{problem}{18.47}
Certain bacteria use the enzyme penicillinase to decompose penicillin and render it inactive. The Michaelis–Menten constants for this enzyme and substrate are $K_m=5\times10^{-5}$ mol L$^{-1}$ and $k_2=2\times10^3s^{-1}$.\\
(a) What is the maximum rate of decomposition of penicillin if the enzyme concentration is $6\times10^{-7}$ \textsc{m}?\\
(b) At what substrate concentration will the rate of decomposition be half that calculated in part (a)?
\end{problem}
\begin{sol}
\\(a) The maximum rate of decomposition of penicillin is
\[
V_m=k_2[E_T]=2\times10^3s^{-1}\times6\times10^{-7}mol~L^{-1}=\uline{1.8\times10^{-3}mol~L^{-1}~s^{-1}}
\]
(b) When the rate of decomposition is half that calculated in part (a)
\begin{gather*}
\frac{V_{\max}[S]}{K_m+[S]}=\frac{1}{2}V_{\max}\Longrightarrow\frac{[S]}{5\times10^{-5}mol~L^{-1}+[S]}=\frac{1}{2}\\
\Longrightarrow[S]=\uline{5\times10^{-5}mol~L^{-1}}
\end{gather*}
\end{sol}

\begin{problem}{18.50}
Suppose $1.00$ L of $9.95\times10^{-3}$ \textsc{m} \ce{S2O3^{2-}} is mixed with $1.00$ L of $2.52\times10^{-3}$ \textsc{m} \ce{H2O2} at a pH of $7.0$ and a temperature of $25^{\circ}$C. These species react by two competing pathways, represented by the balanced equations.
\begin{center}
\ce{S2O3^{2-} + 4H2O2 -> 2SO4^{2-} + H2O + 2H3O+}\\
\ce{2S2O3^{2-} + H2O2 + 2H3O+ -> S4O6^{2-} + 4H2O}
\end{center}
At the instant of mixing, the thiosulfate ion (\ce{S2O3^{2-}}) is observed to be disappearing at the rate of $7.9\times10^{-7}$ mol L$^{-1}$ s$^{-1}$. At the same moment, the \ce{H2O2} is disappearing at the rate of $8.8\times10^{-7}$ mol L$^{-1}$ s$^{-1}$.\\
(a) Compute the percentage of the \ce{S2O3^{2-}} that is, at that moment, reacting according to the first equation.\\
(b) It is observed that the hydronium ion concentration drops. Use the data and answer from part (a) to compute how many milliliters per minute of $0.100$ \textsc{m} \ce{H3O+} must be added to keep the pH equal to $7.0$.
\end{problem}
\begin{sol}
\\(a) Suppose the percentage of the \ce{S2O3^{2-}} that is reacting according to the first equation is $\alpha$
\begin{gather*}
4~\text{rate}_1+\text{rate}_2=-\frac{d[H_2O_2]}{dt}\\
\Longrightarrow4\times7.9\times10^{-7}mol~L^{-1}\alpha+\frac{1}{2}\times7.9\times10^{-7}mol~L^{-1}(1-\alpha)=8.8\times10^{-7}mol~L^{-1}\\
\Longrightarrow\alpha=\uline{18\%}
\end{gather*}
(b) The rate of the two reaction are
\begin{gather*}
\text{rate}_1=7.9\times10^{-7}mol~L^{-1}~s^{-1}\alpha=1.4\times10^{-7}mol~L^{-1}~s^{-1}\\
\text{rate}_2=\frac{1}{2}7.9\times10^{-7}mol~L^{-1}~s^{-1}(1-\alpha)=3.2\times10^{-7}mol~L^{-1}~s^{-1}
\end{gather*}
The rate of hydronium ion concentration dropping is
\[
\frac{d[H_3O^+]}{dt}=2~\text{rate}_1-2\text{rate}_2=-3.6\times10^{-7}mol~L^{-1}~s^{-1}
\]
The number of moles of consumed hydronium in $1$ min is
\[
\Delta n(H_3O^+)=-\frac{d[H_3O^+]}{dt}\times60s\times(1.00L+1.00L)=4.32\times10^{-5}mol
\]
The volume needed to add in $1$ min to keep the pH equal to $7.0$ is
\[
V=\frac{\Delta n(H_3O^+)}{c_0[H_3O^+]}=\frac{4.32\times10^{-5}mol}{0.100mol~L^{-1}}=4.32\times10^{-4}L=\uline{0.432mL}
\]
\end{sol}
\end{document}